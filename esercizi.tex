%%%%%%%%%%%%%%%%%%%%%%%%%%%%%%%%%%%%%%%%%
% Structured General Purpose Assignment
% LaTeX Template
%
% This template has been downloaded from:
% http://www.latextemplates.com
%
% Original author:
% Ted Pavlic (http://www.tedpavlic.com)
%
% Note:
% The \lipsum[#] commands throughout this template generate dummy text
% to fill the template out. These commands should all be removed when 
% writing assignment content.
%
%%%%%%%%%%%%%%%%%%%%%%%%%%%%%%%%%%%%%%%%%

%----------------------------------------------------------------------------------------
%	PACKAGES AND OTHER DOCUMENT CONFIGURATIONS
%----------------------------------------------------------------------------------------

\documentclass{article}

\usepackage{fancyhdr} % Required for custom headers
\usepackage{lastpage} % Required to determine the last page for the footer
\usepackage{extramarks} % Required for headers and footers
\usepackage{graphicx} % Required to insert images
\usepackage{lipsum} % Used for inserting dummy 'Lorem ipsum' text into the template
\usepackage{amsmath}
\usepackage[italian]{babel}
\usepackage[utf8x]{inputenc}
\usepackage{array}
\usepackage[usestackEOL]{stackengine}[2013-10-15]
\def\x{\hspace{3ex}}    %BETWEEN TWO 1-DIGIT NUMBERS
\def\y{\hspace{2.45ex}}  %BETWEEN 1 AND 2 DIGIT NUMBERS
\def\z{\hspace{1.9ex}}    %BETWEEN TWO 2-DIGIT NUMBERS
\stackMath

% Margins
\topmargin=-0.45in
\evensidemargin=0in
\oddsidemargin=0in
\textwidth=6.5in
\textheight=9.0in
\headsep=0.25in 

\linespread{1.1} % Line spacing

% Set up the header and footer
\pagestyle{fancy}
\lhead{\hmwkAuthorName} % Top left header
\chead{\hmwkClass\ (\hmwkClassInstructor\ \hmwkClassTime): \hmwkTitle} % Top center header
\rhead{\firstxmark} % Top right header
\lfoot{\lastxmark} % Bottom left footer
\cfoot{} % Bottom center footer
\rfoot{Page\ \thepage\ of\ \pageref{LastPage}} % Bottom right footer
\renewcommand\headrulewidth{0.4pt} % Size of the header rule
\renewcommand\footrulewidth{0.4pt} % Size of the footer rule

\setlength\parindent{0pt} % Removes all indentation from paragraphs

%----------------------------------------------------------------------------------------
%	DOCUMENT STRUCTURE COMMANDS
%	Skip this unless you know what you're doing
%----------------------------------------------------------------------------------------

% Header and footer for when a page split occurs within a problem environment
\newcommand{\enterProblemHeader}[1]{
\nobreak\extramarks{#1}{#1 continued on next page\ldots}\nobreak
\nobreak\extramarks{#1 (continued)}{#1 continued on next page\ldots}\nobreak
}

% Header and footer for when a page split occurs between problem environments
\newcommand{\exitProblemHeader}[1]{
\nobreak\extramarks{#1 (continued)}{#1 continued on next page\ldots}\nobreak
\nobreak\extramarks{#1}{}\nobreak
}

\setcounter{secnumdepth}{0} % Removes default section numbers
\newcounter{homeworkProblemCounter} % Creates a counter to keep track of the number of problems

\newcommand{\homeworkProblemName}{}
\newenvironment{homeworkProblem}[1][Problem \arabic{homeworkProblemCounter}]{ % Makes a new environment called homeworkProblem which takes 1 argument (custom name) but the default is "Problem #"
\stepcounter{homeworkProblemCounter} % Increase counter for number of problems
\renewcommand{\homeworkProblemName}{#1} % Assign \homeworkProblemName the name of the problem
\section{\homeworkProblemName} % Make a section in the document with the custom problem count
\enterProblemHeader{\homeworkProblemName} % Header and footer within the environment
}{
\exitProblemHeader{\homeworkProblemName} % Header and footer after the environment
}

\newcommand{\problemAnswer}[1]{ % Defines the problem answer command with the content as the only argument
\noindent\framebox[\columnwidth][c]{\begin{minipage}{0.98\columnwidth}#1\end{minipage}} % Makes the box around the problem answer and puts the content inside
}

\newcommand{\homeworkSectionName}{}
\newenvironment{homeworkSection}[1]{ % New environment for sections within homework problems, takes 1 argument - the name of the section
\renewcommand{\homeworkSectionName}{#1} % Assign \homeworkSectionName to the name of the section from the environment argument
\subsection{\homeworkSectionName} % Make a subsection with the custom name of the subsection
\enterProblemHeader{\homeworkProblemName\ [\homeworkSectionName]} % Header and footer within the environment
}{
\enterProblemHeader{\homeworkProblemName} % Header and footer after the environment
}
   
%----------------------------------------------------------------------------------------
%	NAME AND CLASS SECTION
%----------------------------------------------------------------------------------------

\newcommand{\hmwkTitle}{Esercizi} % Assignment title
\newcommand{\hmwkDueDate}{Monday,\ January\ 1,\ 2012} % Due date
\newcommand{\hmwkClass}{Teoria Dei Grafi} % Course/class
\newcommand{\hmwkClassTime}{} % Class/lecture time
\newcommand{\hmwkClassInstructor}{Prof. Ottavio D'Antona} % Teacher/lecturer
\newcommand{\hmwkAuthorName}{Marco Odore} % Your name
\newcommand{\hmwkMatricola}{868906}

%----------------------------------------------------------------------------------------
%	TITLE PAGE
%----------------------------------------------------------------------------------------

\title{
\vspace{2in}
\textmd{\textbf{\hmwkClass:\ \hmwkTitle}}\\
\vspace{0.1in}\large{\textit{\hmwkClassInstructor\ \hmwkClassTime}}
\vspace{3in}
}

\author{\textbf{\hmwkAuthorName\ \hmwkMatricola}}
\date{} % Insert date here if you want it to appear below your name

%----------------------------------------------------------------------------------------

\begin{document}

\maketitle

%----------------------------------------------------------------------------------------
%	TABLE OF CONTENTS
%----------------------------------------------------------------------------------------

%\setcounter{tocdepth}{1} % Uncomment this line if you don't want subsections listed in the ToC

\newpage
\tableofcontents
\newpage

%----------------------------------------------------------------------------------------
%	PROBLEM 1
%----------------------------------------------------------------------------------------

% To have just one problem per page, simply put a \clearpage after each problem

\begin{homeworkProblem}[Esercizio 1.3] 
Di quanti elementi è costituito l'insieme potenza dell'insieme vuoto? E del singoletto \{a\}?
\vspace{10pt} % Question

\problemAnswer{ % Answer
Dato che la cardinalità dell'insieme potenza di un insieme S è pari a
\[
|P(S)| = 2^{|S|}
\]
la cardinalità dell'insieme potenza dell'insieme vuoto è
\[
2^{0} = 1
\]
(l'insieme potenza contiene solo l'insieme vuoto)
\newline
mentre la cardinalità dell'insieme potenza dell'insieme singoletto \{a\} è pari a
\[
2^{1} = 2
\]
(l'insieme potenza contiene l'insieme vuoto e il singoletto)
}
\end{homeworkProblem}

%----------------------------------------------------------------------------------------
%	PROBLEM 2
%----------------------------------------------------------------------------------------

\begin{homeworkProblem}[Esercizio 1.5] % Custom section title
Quanto vale S(n,2)? Quanto vale S(n, n-1)?

%--------------------------------------------

\problemAnswer{
(i) Calcolando il caso base S(2,2) otteniamo
\[
S(2,2) = 1 = 2 ^{1}-1
\]
(Il numero di Stirling di seconda specie ha i casi speciali S(n,n) = S(n,1) = 1)
\newline
il caso S(3,2)
\[
S(3,2) = S(2,1) + 2S(2,2) = 1 + 2 = 3 = 2^{2}-1
\]
(dato che S(n,k) può essere calcolato come S(n-1,k-1) + kS(n-1,k)))
\newline
il caso S(4,2)
\[
S(4,2) = S(3,1) + 2S(3,2) = 1 + 6 = 7 = 2^{3}-1
\]
etc, possiamo affermare che S(n,2) è pari a 
\[
2^{n-1}-1
\]
\newline
(ii) Partendo dal caso base S(2,1) otteniamo
\[
S(2,1) = 1
\]
il caso S(3,2)
\[
S(3,2) = S(2,1) + 2S(2,2) = 1 + 2 = 3 
\]
il caso S(4,3)
\[
S(4,3) = S(3,2) + 3S(3,3) = 3 + 3 = 6
\]
etc, possiamo dedurre che nel caso S(n,n-1), il numero di stirling è pari a
\[
\sum_{i=1}^{n-1} i
\]
}
--------------------------

\end{homeworkProblem}

%----------------------------------------------------------------------------------------
%	PROBLEM 3
%----------------------------------------------------------------------------------------

\begin{homeworkProblem}[Esercizio 1.6] % Roman numerals
Verificare che la suddivisione dei sottoinsiemi di un insieme S a seconda del numero di elementi che essi contengono costituisce una partizione dell'insieme potenza di S.

\problemAnswer{
Dato un insieme esempio S = \{a, b, c\}, il suo insieme delle parti è dato da
\[
\{a,b,c\}, \{a,b\}, \{a,c\}, \{b,c\}, \{a\}, \{b\}, \{c\}, 0
\]
Se raggruppiamo i sottoinsiemi a seconda del numero di elementi che essi contengono otteniamo:
\[
|\{a,b,c\}|, |\{a,b\}, \{a,c\}, \{b,c\}|, |\{a\}, \{b\}|, |\{c\}|, |0|
\]
che rappresenta esattamente una partizione dell'insieme delle parti.
}

%--------------------------------------------

\end{homeworkProblem}

%----------------------------------------------------------------------------------------
%	PROBLEM 4
%----------------------------------------------------------------------------------------

\begin{homeworkProblem}[Esercizio 1.7] % Roman numerals
Quanto vale la somma degli elementi della n-esima riga del triangolo di Tartaglia?

\problemAnswer{
Sapendo che ogni elemento di una riga del triangolo di tartaglia è calcolabile come il binomiale
\[
{n \choose k}
\]
dove n è la riga in oggetto, e k è la posizione nella riga corrente, possiamo calcolare la somma degli elementi della n-esemia riga del triangolo di tartaglia come
\[
\sum_{k=0}^{n}{n \choose k}
\]
}
\end{homeworkProblem}

\begin{homeworkProblem}[Esercizio 1.8]
Che relazione c'è tra la somma degli elementi di posto pari e quelli di posto dispari nelle righe del triangolo di Tartaglia? In che senso la soluzione a questo esercizio suggerisce un criterio di divisibilità tra polinomi?

\problemAnswer{
Dato il triangolo di tartaglia
\begin{center}
\begin{tabular}{rccccccccc}
$n=0$:&    &    &    &    &  1\\\noalign{\smallskip\smallskip}
$n=1$:&    &    &    &  1 &    &  1\\\noalign{\smallskip\smallskip}
$n=2$:&    &    &  1 &    &  2 &    &  1\\\noalign{\smallskip\smallskip}
$n=3$:&    &  1 &    &  3 &    &  3 &    &  1\\\noalign{\smallskip\smallskip}
$n=4$:&  1 &    &  4 &    &  6 &    &  4 &    &  1\\\noalign{\smallskip\smallskip}
\end{tabular}
\end{center}
partendo dalla seconda riga e sommando gli elementi di posto pari e di posto dispari otteniamo le coppie
\begin{gather}
(1, 1)  \quad n = 1\\
(2, 2)  \quad n = 2\\
(4, 4)  \quad n = 3\\
(8, 8)  \quad n = 4\\
etc
\end{gather}
. Notiamo che le somme per pari e dispari sono identiche per riga (oltre a rappresantare le potenze di 2).
\newline
Sviluppando poi i polinomi $(x-1)^n$ al variare di n, otteniamo
\begin{gather}
(x-1)^1 = x - 1\\
(x-1)^2 = x^2 -2x + 2\\
(x-1)^3 = x^3 -3x + 3x -1\\ 
(x-1)^4 = x^4 -4x^3 + 6x^2 - 4x +1\\
etc
\end{gather}
e cioè polinomi i cui coefficienti sono gli elementi delle righe del triangolo di Tartaglia (che variano al variare di n). Possiamo inoltre notare che tali coefficienti hanno i segni alternati in base alle posizione (pari positivi, dispari negativi). 
\newline
Secondo il teorema di Ruffini un polinomio è divisibile per il binomio (x-k) se e solo se $p(k) = 0$. Secondo il teorema del Resto, p(k) non è altro che il resto della divisione tra un polinomio e il binomio (x-k). Dato che nel nostro caso k=1, possiamo verificare che $p(1) = 0$ per ogni $p = (x-1)^n$, dato che mettere x a 1 corrisponde a sommare i coefficienti del polinomio, che come abbiamo verificando in partenza corrispondo ai coefficienti della riga n-esima del triangolo di tartaglia, alternati di segno per la posizione, e che quindi sommati portano a 0.
}
\end{homeworkProblem}
\begin{homeworkProblem}[Esercizio 1.9]
Dimostrare che, presi comunque $a,b,n \in N$
\[
(a+\sqrt{b})^n + (a-\sqrt{b})^n \in N
\]
\problemAnswer{
Sapendo che il binomio di Newton è calcolato come
\[
(a+b)^n = \sum_{k=0}^{n}{n \choose k} a^{n-k}\cdot b^k
\]
Svolgendo singolarmente $(a+\sqrt{b})^n$ e $(a-\sqrt{b})^n$ otteniamo
\[
(a+\sqrt{b})^n = \sum_{k=0}^{n}{n \choose k} a^{n-k}\cdot (\sqrt{b})^{k}
\]
\[
(a-\sqrt{b})^n = \sum_{k=0}^{n}{n \choose k} a^{n-k}\cdot (-\sqrt{b})^{k}
\]
quando k è pari il termine k-esimo dei due polinomi coincide (il segno negativo diventa positivo per il secondo termine del secondo binomio)
\[
(\sqrt{b})^{k} = (-\sqrt{b})^{k} = b^{k/2} \in N 
\]
poichè $k/2 \in N$.
Mentre quando k è dispari abbiamo che  
\[
(\sqrt{b})^{k} +(-\sqrt{b})^{k} = 0
\]
che quindi porta all'eliminazione del termine k-esimo all'interno della sommatoria (i valori k-esimi sono i medesimi per i due binomi, ma opposti per segno).
\newline
Dato che $n-k \in N$
\[
{n \choose k} \in N
\]
\[
a^{n-k} \in N
\]
Possiamo quindi affermare che comunque scelti $a,b,n \in N$, il termine k-esimo delle sommatoria (con k pari)
\[
{n \choose k} a^{n-k}\cdot (\sqrt{b})^{k} \in N
\]
e di conseguenza che
\[
(a+\sqrt{b})^n + (a-\sqrt{b})^n \in N
\]
}
\end{homeworkProblem}
\begin{homeworkProblem}[Esercizio 1.23]
Dimostrare che l'elemento di posto (i, j) del prodotto di n matrici triangolari inferiori (di ordine arbitrario) è dato dalla somma di
\[
\frac{<n>_{i-j}}{(i-j)!} = \left({n \choose i-j}\right)
\]
addendi.

\problemAnswer{
Prima di tutto è importante ricordare una proprietà del valore $\frac{<n>_{i-j}}{(i-j)!}$, che può essere rappresentato come il binomiale
\[
{n+(i-j)-1 \choose i-j} = {n+(i-j)-1 \choose n-1}
\]
Inoltre data una matrice triangolare inferiore, il valore dell'elemento della matrice potenza $N^{n}$ in posizione $(i,j)$ è dato dalla formula ricorsiva:
\[
(i,j)^{n}=\sum_{k=0}^{(i-j)} (a_{ik}^{n-1}\cdot a_{kj})
\]
\begin{footnotesize}
\textbf{Nota:} $(i,j)^{n}$ e $a_{ik}^{n-1}$ vanno intesoi come elementi della matrice $N^{n}$ e $N^{n-1}$, non come elementi elevati alla potenza n-esima
\end{footnotesize}

Proviamo a calcolare la matrice N, risultato del seguente prodotto di matrici triangolari inferiori
\[
N =
\left[
\begin{array}{ccc}
1 & 0 & 0 \\
1 & 1 & 0 \\
1 & 1 & 1 \end{array} \right] 
\cdot
\left[
\begin{array}{ccc}
1 & 0 & 0 \\
1 & 1 & 0 \\
1 & 1 & 1 \end{array} \right]
\cdot
\left[
\begin{array}{ccc}
1 & 0 & 0 \\
1 & 1 & 0 \\
1 & 1 & 1 \end{array} \right] = 
\left[
\begin{array}{ccc}
1 & 0 & 0 \\
3 & 1 & 0 \\
6 & 3 & 1 \end{array} \right]
\]
che può essere scritta anche come il prodotto
\[
N =
\left[
\begin{array}{ccc}
1 & 0 & 0 \\
2 & 1 & 0 \\
3 & 2 & 1 \end{array} \right]
\cdot
\left[
\begin{array}{ccc}
1 & 0 & 0 \\
1 & 1 & 0 \\
1 & 1 & 1 \end{array} \right] = 
\left[
\begin{array}{ccc}
1 & 0 & 0 \\
3 & 1 & 0 \\
6 & 3 & 1 \end{array} \right]
\]
dato che la matrice parziale P è uguale a 
\[
P =
\left[
\begin{array}{ccc}
1 & 0 & 0 \\
1 & 1 & 0 \\
1 & 1 & 1 \end{array} \right] 
\cdot
\left[
\begin{array}{ccc}
1 & 0 & 0 \\
1 & 1 & 0 \\
1 & 1 & 1 \end{array} \right]
=
\left[
\begin{array}{ccc}
1 & 0 & 0 \\
2 & 1 & 0 \\
3 & 2 & 1 \end{array} \right]
\]
. Calcolando passo passo i valori degli elementi in posizione (i=2,j=0), (i=2, j=1), (i=2, j=2) della matrice P, otteniamo
\[
(i=2, j=0) =  ( 1 \cdot 1 ) + ( 1 \cdot 1 ) + ( 1 \cdot 1 ) = 3 (addendi)
\]
\[
(i=2, j=1) =  ( 1 \cdot 0 ) + ( 1 \cdot 1 ) + ( 1 \cdot 1 ) = 2 (addendi)
\]
\[
(i=2, j=2) =  ( 1 \cdot 0 ) + ( 1 \cdot 0 ) + ( 1 \cdot 1 ) = 1 (addendi)
\]
che effettivamente corrispondono ai valori calcolati con il binomiale (n=2)
\[
{n+(i-j)-1 \choose n-1}
\]
\[
(i=2, j= 0) = {2+(2-0)-1 \choose 2-1} = {3 \choose 1} = 3 
\]
\[
(i=2, j= 1) = {2+(2-1)-1 \choose 2-1} = {2 \choose 1} = 2 
\]
\[
(i=2, j= 2) = {2+(2-2)-1 \choose 2-1} = {1 \choose 1} = 1
\]
}
\problemAnswer{
Calcolando ora i valori degli elementi in posizione (i=2,j=0), (i=2, j=1), (i=2, j=2) della matrice N, ricordandoci del calcolo fatto precedentemente con P, otteniamo
\[
(i=2, j=0) =  ((( 1 \cdot 1 ) + ( 1 \cdot 1 ) + ( 1 \cdot 1 )) \cdot 1) + ((( 1 \cdot 0 ) + ( 1 \cdot 1 ) + ( 1 \cdot 1 )) \cdot 1) + ((( 1 \cdot 0 ) + ( 1 \cdot 0 ) + ( 1 \cdot 1 )) \cdot 1)  = 3 + 2 + 1 = 6 
\]
addendi
\[
(i=2, j=1) =  ((( 1 \cdot 1 ) + ( 1 \cdot 1 ) + ( 1 \cdot 1 )) \cdot 0 ) + ((( 1 \cdot 0 ) + ( 1 \cdot 1 ) + ( 1 \cdot 1 )) \cdot 1 ) + ((( 1 \cdot 0 ) + ( 1 \cdot 0 ) + ( 1 \cdot 1 )) \cdot 1 ) = 2 + 1 = 3
\]
addendi
\[
(i=2, j=2) =  ((( 1 \cdot 1 ) + ( 1 \cdot 1 ) + ( 1 \cdot 1 )) \cdot 0 ) + ((( 1 \cdot 0 ) + ( 1 \cdot 1 ) + ( 1 \cdot 1 )) \cdot 0 ) + ((( 1 \cdot 0 ) + ( 1 \cdot 0 ) + ( 1 \cdot 1 )) \cdot 1 ) =  1 = 1
\]
addendi.
Che effettivamente corrispondono ai valori calcolati con il binomiale (n=3)
\[
(i=2, j= 0) = {3+(2-0)-1 \choose 3-1} = {4 \choose 2} = 6 
\]
\[
(i=2, j= 1) = {3+(2-1)-1 \choose 3-1} = {3 \choose 2} = 3 
\]
\[
(i=2, j= 2) = {3+(2-2)-1 \choose 3-1} = {2 \choose 2} = 1
\]
}
\end{homeworkProblem}
\begin{homeworkProblem}[Esercizio 1.25]
Dimostrare che
\[
P(n, x) = P(n-1, x-1) + P(n-x, x)
\]
sapendo che 
\[
P(1,1) = 1 
\]
e
\[
P(n,x) = 0 
\]
se n $\leq$ 0

\problemAnswer{
Ricordando che $P(n,x)$ non rappresenta altro il numero di modi in cui possiamo ottenere il numero $n$ sommando $x$ numeri naturali, dimostriamo questa equazione per induzione, partendo dal caso base $P(2,1)$, che sappiamo valere 1 (2 è rappresentabile in un solo modo usando una singola cifra): 
\[
P(2,1) = P(1,0) + P(1,1) = 0 + 1 = 1
\]
($P(1,0)=0$ in quanto non esistono modi per rappresentare 1 con 0 cifre)
Verifichiamolo ora per il caso n+1
}
\end{homeworkProblem}
\begin{homeworkProblem}[Esercizio "Nuovo"]
Dimostrare che 
\[
s(n,1) =(n-1)!
\]
\problemAnswer{
Sappiamo che il fattoriale di $(n-1)$ si può scrivere anche come
\[
(n-1)! = (n-1) \cdot (n-2) \cdot ... \cdot 1
\]
e che generalmente
\[
s(n,k) = s(n-1, k-1) + (n-1) \cdot s(n-1,k)
\]
Svolgendo $s(n,1)$ otteniamo
\[
s(n,1) = s(n-1, 0) + (n-1) \cdot s(n-1, 1)
\]
Ma dato che 
\[
s(n-1, 0) = 0
\]
Possiamo affermare quindi che
\[
s(n,1) = (n-1)\cdot s(n-1,1)
\]
e conseguentemente
\[
s(n-1, 1) = (n-2)\cdot s(n-2, 1)
\]
Calcolando quindi ricorsivamente $s(n-i, 1)$ fino a che $n-i =1$ (arrivando al caso $s(1,1)=1$ )otteniamo
\[
(n-1)\cdot(n-2)\cdot...\cdot 1 = (n-1)!
\]
}
\end{homeworkProblem}%----------------------------------------------------------------------------------------

\end{document}
