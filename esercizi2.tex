%%%%%%%%%%%%%%%%%%%%%%%%%%%%%%%%%%%%%%%%%
% Structured General Purpose Assignment
% LaTeX Template
%
% This template has been downloaded from:
% http://www.latextemplates.com
%
% Original author:
% Ted Pavlic (http://www.tedpavlic.com)
%
% Note:
% The \lipsum[#] commands throughout this template generate dummy text
% to fill the template out. These commands should all be removed when 
% writing assignment content.
%
%%%%%%%%%%%%%%%%%%%%%%%%%%%%%%%%%%%%%%%%%

%----------------------------------------------------------------------------------------
%	PACKAGES AND OTHER DOCUMENT CONFIGURATIONS
%----------------------------------------------------------------------------------------

\documentclass{article}

\usepackage{fancyhdr} % Required for custom headers
\usepackage{lastpage} % Required to determine the last page for the footer
\usepackage{extramarks} % Required for headers and footers
\usepackage{graphicx} % Required to insert images
\usepackage{lipsum} % Used for inserting dummy 'Lorem ipsum' text into the template
\usepackage{amsmath}
\usepackage[italian]{babel}
\usepackage[utf8x]{inputenc}
\usepackage{array}
\usepackage{float}
\usepackage{multirow}
\usepackage[usestackEOL]{stackengine}[2013-10-15]
\def\x{\hspace{3ex}}    %BETWEEN TWO 1-DIGIT NUMBERS
\def\y{\hspace{2.45ex}}  %BETWEEN 1 AND 2 DIGIT NUMBERS
\def\z{\hspace{1.9ex}}    %BETWEEN TWO 2-DIGIT NUMBERS
\stackMath

% Margins
\topmargin=-0.45in
\evensidemargin=0in
\oddsidemargin=0in
\textwidth=6.5in
\textheight=9.0in
\headsep=0.25in 

\linespread{1.1} % Line spacing

% Set up the header and footer
\pagestyle{fancy}
\lhead{\hmwkAuthorName} % Top left header
\chead{\hmwkClass\: \hmwkTitle} % Top center header
\rhead{\firstxmark} % Top right header
\lfoot{\lastxmark} % Bottom left footer
\cfoot{} % Bottom center footer
\rfoot{Page\ \thepage\ of\ \pageref{LastPage}} % Bottom right footer
\renewcommand\headrulewidth{0.4pt} % Size of the header rule
\renewcommand\footrulewidth{0.4pt} % Size of the footer rule

\setlength\parindent{0pt} % Removes all indentation from paragraphs

%----------------------------------------------------------------------------------------
%	DOCUMENT STRUCTURE COMMANDS
%	Skip this unless you know what you're doing
%----------------------------------------------------------------------------------------

% Header and footer for when a page split occurs within a problem environment
\newcommand{\enterProblemHeader}[1]{
\nobreak\extramarks{#1}{#1 continued on next page\ldots}\nobreak
\nobreak\extramarks{#1 (continued)}{#1 continued on next page\ldots}\nobreak
}

% Header and footer for when a page split occurs between problem environments
\newcommand{\exitProblemHeader}[1]{
\nobreak\extramarks{#1 (continued)}{#1 continued on next page\ldots}\nobreak
\nobreak\extramarks{#1}{}\nobreak
}

\setcounter{secnumdepth}{0} % Removes default section numbers
\newcounter{homeworkProblemCounter} % Creates a counter to keep track of the number of problems

\newcommand{\homeworkProblemName}{}
\newenvironment{homeworkProblem}[1][Problem \arabic{homeworkProblemCounter}]{ % Makes a new environment called homeworkProblem which takes 1 argument (custom name) but the default is "Problem #"
\stepcounter{homeworkProblemCounter} % Increase counter for number of problems
\renewcommand{\homeworkProblemName}{#1} % Assign \homeworkProblemName the name of the problem
\section{\homeworkProblemName} % Make a section in the document with the custom problem count
\enterProblemHeader{\homeworkProblemName} % Header and footer within the environment
}{
\exitProblemHeader{\homeworkProblemName} % Header and footer after the environment
}

\newcommand{\problemAnswer}[1]{ % Defines the problem answer command with the content as the only argument
\noindent\framebox[\columnwidth][c]{\begin{minipage}{0.98\columnwidth}#1\end{minipage}} % Makes the box around the problem answer and puts the content inside
}

\newcommand{\homeworkSectionName}{}
\newenvironment{homeworkSection}[1]{ % New environment for sections within homework problems, takes 1 argument - the name of the section
\renewcommand{\homeworkSectionName}{#1} % Assign \homeworkSectionName to the name of the section from the environment argument
\subsection{\homeworkSectionName} % Make a subsection with the custom name of the subsection
\enterProblemHeader{\homeworkProblemName\ [\homeworkSectionName]} % Header and footer within the environment
}{
\enterProblemHeader{\homeworkProblemName} % Header and footer after the environment
}
   
%----------------------------------------------------------------------------------------
%	NAME AND CLASS SECTION
%----------------------------------------------------------------------------------------

\newcommand{\hmwkTitle}{Esercizi (Insiemi Stabili)} % Assignment title
\newcommand{\hmwkDueDate}{Monday,\ January\ 1,\ 2012} % Due date
\newcommand{\hmwkClass}{Teoria Dei Grafi} % Course/class
\newcommand{\hmwkClassTime}{} % Class/lecture time
\newcommand{\hmwkClassInstructor}{Prof. Ottavio D'Antona} % Teacher/lecturer
\newcommand{\hmwkAuthorName}{Marco Odore} % Your name
\newcommand{\hmwkMatricola}{868906}

%----------------------------------------------------------------------------------------
%	TITLE PAGE
%----------------------------------------------------------------------------------------

\title{
\vspace{2in}
\textmd{\textbf{\hmwkClass:\ \hmwkTitle}}\\
\vspace{0.1in}\large{\textit{\hmwkClassInstructor\ \hmwkClassTime}}
\vspace{3in}
}

\author{\textbf{\hmwkAuthorName\ \hmwkMatricola}}
\date{} % Insert date here if you want it to appear below your name

%----------------------------------------------------------------------------------------

\begin{document}

\maketitle

%----------------------------------------------------------------------------------------
%	TABLE OF CONTENTS
%----------------------------------------------------------------------------------------

%\setcounter{tocdepth}{1} % Uncomment this line if you don't want subsections listed in the ToC

\newpage
\tableofcontents
\newpage

%----------------------------------------------------------------------------------------
%	PROBLEM 1
%----------------------------------------------------------------------------------------

% To have just one problem per page, simply put a \clearpage after each problem

\begin{homeworkProblem}[Esercizio 1] 
Dimostrare che il numero di insiemi stabili di un cammino $P_n$ (dove $n$ indica il numero di vertici del cammino) ha la ricorrenza di Fibonacci
\[
P_n = P_{n-1} + P_{n-2}
\]
\vspace{10pt} % Question

\problemAnswer{ % Answer
}
\end{homeworkProblem}

\begin{homeworkProblem}[Esercizio 1.1] 
Tabella degli insiemi stabili di un cammino $P_{n,k}$ al variare della lunghezza del cammino $n$ e del numero $k$ di vertici non adiacenti.

\vspace{10pt} % Question

\problemAnswer{ % Answer
% Please add the following required packages to your document preamble:
% \usepackage{multirow}
\begin{table}[H]
\centering
\begin{tabular}{l|l|ccccccccc}
$P_{n,k}$             & \multicolumn{10}{c|}{$k$}                                                                                                                                                                                                                              \\ \hline
\multirow{10}{*}{$n$} & \multicolumn{1}{c|}{} & \multicolumn{1}{c|}{0} & \multicolumn{1}{c|}{1} & \multicolumn{1}{c|}{2} & \multicolumn{1}{c|}{3} & \multicolumn{1}{c|}{4} & \multicolumn{1}{c|}{5} & \multicolumn{1}{c|}{6} & \multicolumn{1}{c|}{7} & \multicolumn{1}{c|}{8} \\ \cline{2-11} 
                      & 0                     & 1                      &                        &                        &                        &                        &                        &                        &                        &                        \\ \cline{2-2}
                      & 1                     & 1                      & 1                      &                        &                        &                        &                        &                        &                        &                        \\ \cline{2-2}
                      & 2                     & 1                      & 2                      &                        &                        &                        &                        &                        &                        &                        \\ \cline{2-2}
                      & 3                     & 1                      & 3                      & 1                      &                        &                        &                        &                        &                        &                        \\ \cline{2-2}
                      & 4                     & 1                      & 4                      & 3                      &                        &                        &                        &                        &                        &                        \\ \cline{2-2}
                      & 5                     & 1                      & 5                      & 6                      & 1                      &                        &                        &                        &                        &                        \\ \cline{2-2}
                      & 6                     & 1                      & 6                      & 10                     & 4                      &                        &                        &                        &                        &                        \\ \cline{2-2}
                      & 7                     & 1                      & 7                      &                        &                        &                        &                        &                        &                        &                        \\ \cline{2-2}
                      & 8                     & 1                      & 8                      &                        &                        &                        &                        &                        &                        &                        \\ \cline{1-2}
\end{tabular}
\caption{My caption}
\label{my-label}
\end{table}

}
\end{homeworkProblem}

\begin{homeworkProblem}[Esercizio 1.2] 
Scrivere un algoritmo capace di enumerare gli insiemi stabili di un grafo arbitrario.

\vspace{10pt} % Question

\problemAnswer{ % Answer
}
\end{homeworkProblem}
\begin{homeworkProblem}[Esercizio 2] 
Tabella degli insiemi stabili di un ciclo $C_{n,k}$ al variare della lunghezza del ciclo $n$ e del numero $k$ di vertici non adiacenti.
\vspace{10pt} % Question

\problemAnswer{ % Answer
}
\end{homeworkProblem}

\begin{homeworkProblem}[Esercizio 2.1] 
Trovare la formula di ricorrenza di un ciclo $C_{n, k}$ al variare della lunghezza del ciclo $n$ e del numero $k$ di vertici non adiacenti.
\vspace{10pt} % Question

\problemAnswer{ % Answer
}
\end{homeworkProblem}
%----------------------------------------------------------------------------------------

\end{document}
