%%%%%%%%%%%%%%%%%%%%%%%%%%%%%%%%%%%%%%%%%
% Structured General Purpose Assignment
% LaTeX Template
%
% This template has been downloaded from:
% http://www.latextemplates.com
%
% Original author:
% Ted Pavlic (http://www.tedpavlic.com)
%
% Note:
% The \lipsum[#] commands throughout this template generate dummy text
% to fill the template out. These commands should all be removed when 
% writing assignment content.
%
%%%%%%%%%%%%%%%%%%%%%%%%%%%%%%%%%%%%%%%%%

%----------------------------------------------------------------------------------------
%	PACKAGES AND OTHER DOCUMENT CONFIGURATIONS
%----------------------------------------------------------------------------------------

\documentclass{article}

\usepackage{fancyhdr} % Required for custom headers
\usepackage{lastpage} % Required to determine the last page for the footer
\usepackage{extramarks} % Required for headers and footers
\usepackage{graphicx} % Required to insert images
\usepackage{lipsum} % Used for inserting dummy 'Lorem ipsum' text into the template
\usepackage{amsmath}
\usepackage{algorithm2e}
\usepackage[italian]{babel}
\usepackage[utf8x]{inputenc}
\usepackage{array}
\usepackage{float}
\usepackage{tikz}
\usepackage{multirow}
\usepackage[usestackEOL]{stackengine}[2013-10-15]
\def\x{\hspace{3ex}}    %BETWEEN TWO 1-DIGIT NUMBERS
\def\y{\hspace{2.45ex}}  %BETWEEN 1 AND 2 DIGIT NUMBERS
\def\z{\hspace{1.9ex}}    %BETWEEN TWO 2-DIGIT NUMBERS
\stackMath

% Margins
\topmargin=-0.45in
\evensidemargin=0in
\oddsidemargin=0in
\textwidth=6.5in
\textheight=9.0in
\headsep=0.25in 

\linespread{1.1} % Line spacing

% Set up the header and footer
\pagestyle{fancy}
\lhead{\hmwkAuthorName} % Top left header
\chead{\hmwkClass\: \hmwkTitle} % Top center header
\rhead{\firstxmark} % Top right header
\lfoot{\lastxmark} % Bottom left footer
\cfoot{} % Bottom center footer
\rfoot{Page\ \thepage\ of\ \pageref{LastPage}} % Bottom right footer
\renewcommand\headrulewidth{0.4pt} % Size of the header rule
\renewcommand\footrulewidth{0.4pt} % Size of the footer rule

\setlength\parindent{0pt} % Removes all indentation from paragraphs

%----------------------------------------------------------------------------------------
%	DOCUMENT STRUCTURE COMMANDS
%	Skip this unless you know what you're doing
%----------------------------------------------------------------------------------------

% Header and footer for when a page split occurs within a problem environment
\newcommand{\enterProblemHeader}[1]{
\nobreak\extramarks{#1}{#1 continued on next page\ldots}\nobreak
\nobreak\extramarks{#1 (continued)}{#1 continued on next page\ldots}\nobreak
}

% Header and footer for when a page split occurs between problem environments
\newcommand{\exitProblemHeader}[1]{
\nobreak\extramarks{#1 (continued)}{#1 continued on next page\ldots}\nobreak
\nobreak\extramarks{#1}{}\nobreak
}

\setcounter{secnumdepth}{0} % Removes default section numbers
\newcounter{homeworkProblemCounter} % Creates a counter to keep track of the number of problems

\newcommand{\homeworkProblemName}{}
\newenvironment{homeworkProblem}[1][Problem \arabic{homeworkProblemCounter}]{ % Makes a new environment called homeworkProblem which takes 1 argument (custom name) but the default is "Problem #"
\stepcounter{homeworkProblemCounter} % Increase counter for number of problems
\renewcommand{\homeworkProblemName}{#1} % Assign \homeworkProblemName the name of the problem
\section{\homeworkProblemName} % Make a section in the document with the custom problem count
\enterProblemHeader{\homeworkProblemName} % Header and footer within the environment
}{
\exitProblemHeader{\homeworkProblemName} % Header and footer after the environment
}

\newcommand{\problemAnswer}[1]{ % Defines the problem answer command with the content as the only argument
\noindent\framebox[\columnwidth][c]{\begin{minipage}{0.98\columnwidth}#1\end{minipage}} % Makes the box around the problem answer and puts the content inside
}

\newcommand{\homeworkSectionName}{}
\newenvironment{homeworkSection}[1]{ % New environment for sections within homework problems, takes 1 argument - the name of the section
\renewcommand{\homeworkSectionName}{#1} % Assign \homeworkSectionName to the name of the section from the environment argument
\subsection{\homeworkSectionName} % Make a subsection with the custom name of the subsection
\enterProblemHeader{\homeworkProblemName\ [\homeworkSectionName]} % Header and footer within the environment
}{
\enterProblemHeader{\homeworkProblemName} % Header and footer after the environment
}
   
%----------------------------------------------------------------------------------------
%	NAME AND CLASS SECTION
%----------------------------------------------------------------------------------------

\newcommand{\hmwkTitle}{Esercizi (Insiemi Stabili)} % Assignment title
\newcommand{\hmwkDueDate}{Monday,\ January\ 1,\ 2012} % Due date
\newcommand{\hmwkClass}{Teoria Dei Grafi} % Course/class
\newcommand{\hmwkClassTime}{} % Class/lecture time
\newcommand{\hmwkClassInstructor}{Prof. Ottavio D'Antona} % Teacher/lecturer
\newcommand{\hmwkAuthorName}{Marco Odore} % Your name
\newcommand{\hmwkMatricola}{868906}

%----------------------------------------------------------------------------------------
%	TITLE PAGE
%----------------------------------------------------------------------------------------

\title{
\vspace{2in}
\textmd{\textbf{\hmwkClass:\ \hmwkTitle}}\\
\vspace{0.1in}\large{\textit{\hmwkClassInstructor\ \hmwkClassTime}}
\vspace{3in}
}

\author{\textbf{\hmwkAuthorName\ \hmwkMatricola}}
\date{} % Insert date here if you want it to appear below your name

%----------------------------------------------------------------------------------------

\begin{document}

\maketitle

%----------------------------------------------------------------------------------------
%	TABLE OF CONTENTS
%----------------------------------------------------------------------------------------

%\setcounter{tocdepth}{1} % Uncomment this line if you don't want subsections listed in the ToC

\newpage
\tableofcontents
\newpage

%----------------------------------------------------------------------------------------
%	PROBLEM 1
%----------------------------------------------------------------------------------------

% To have just one problem per page, simply put a \clearpage after each problem

\begin{homeworkProblem}[Esercizio 1] 
Dimostrare che il numero di insiemi stabili $S$ di un cammino $P_{n}$ (dove $n$ indica il numero di vertici del cammino) ha la ricorrenza di \emph{Fibonacci}
\[
S(P_n) = S(P_{n-1}) + S(P_{n-2})
\]
\vspace{10pt} % Question

\problemAnswer{ % Answer
Per dimostrare la relazione tra il numero di set indipendenti di un cammino e la sequenza di \emph{Fibonacci}, basta verificare cosa accade all'aggiungere di un vertice ad un cammino $P_n$. Ad esempio dato il seguente cammino:
\begin{center}
\begin{tikzpicture}

  \node[circle,fill=blue!20,draw, minimum size=6mm] (a) at (0, 0) {a};
  \node[circle,fill=blue!20,draw, minimum size=6mm] (b) at (2, 1) {b};
  \node[circle,fill=blue!20,draw, minimum size=6mm] (c) at (4, -0.5) {c};

\draw[-] (a) -- (b) -- (c);
%\draw[dashed] (A) -- (C);
\end{tikzpicture}
\end{center}
Sappiamo che esistono 5 insiemi stabili, che sono dati dagli insiemi singoletto $\{a\}, \{b\}, \{c\}$, dall'insieme $\{a,c\}$ e dall'insieme vuoto.
Cosa accadrebbe se aggiungessimo un nodo d? Avremmo il seguente cammino:
\begin{center}
\begin{tikzpicture}

  \node[circle,fill=blue!20,draw, minimum size=6mm] (a) at (0, 0) {a};
  \node[circle,fill=blue!20,draw, minimum size=6mm] (b) at (2, 1) {b};
  \node[circle,fill=blue!20,draw, minimum size=6mm] (c) at (4, -0.5) {c};
  \node[circle,fill=blue!20,draw, minimum size=6mm] (d) at (6, 0) {d};

\draw[-] (a) -- (b) -- (c) -- (d);
%\draw[dashed] (A) -- (C);
\end{tikzpicture}
\end{center}
Sicuramente i set indipendenti precedentemente individuati con solo $a$, $b$, $c$ rimarrebbero. Quindi c'è qualcosa in più rispetto al cammino precedente. Ma cosa? Osservando il cammino ci rendiamo conto che il nodo aggiunto può generare set indipendenti con solo i nodi che non ha adiacenti, e cioè $a$, $b$. Ma quanti nuovi set indipendenti può generare ? Ne può generare tanti quanti sono i set generabili dal cammino composto solo da $a$, $b$. Questo perché per generarli posso aggiungere l'elemento $d$ ad ogni set presente in quest'ultimo cammino, e cioè $\{a, d\}$, $\{b, d\}$, $\{d\}$, dato che il cammino $a$ $b$ ha gli insiemi $\{a\}$, $\{b\}$ e l'insieme vuoto, come set indipendenti. Quindi possiamo dire che il numero di set indipententi di un cammino generico $P_n$ si può ottenre come somma dei set indipendenti ottenuti dal cammino meno un nodo, e dai set indipendenti ottenuti dal cammino meno due nodi
\[
S(P_n) = S(P_{n-1}) + S(P_{n-2})
\]
Che è effettivamente la ricorrenza di \emph{Fibonacci}.
}
\end{homeworkProblem}


\begin{homeworkProblem}[Esercizio 1.1] 
Tabella degli insiemi stabili di un cammino $P_{n,k}$ al variare della lunghezza del cammino $n$ e del numero $k$ di vertici non adiacenti.

\vspace{10pt} % Question

\problemAnswer{ % Answer
Per calcolare il numero di insiemi stabili con $k$ vertici di un cammino di lunghezza $n$ arbitraria, possiamo fare alcune osservazioni. Prendendo ad esempio il cammino $P_n$ seguente
\begin{center}
\begin{tikzpicture}

  \node[circle,fill=blue!20,draw, minimum size=6mm] (a) at (0, 0) {a};
  \node[circle,fill=blue!20,draw, minimum size=6mm] (b) at (2, 1) {b};
  \node[circle,fill=blue!20,draw, minimum size=6mm] (c) at (4, -0.5) {c};
  \node[circle,fill=blue!20,draw, minimum size=6mm] (d) at (6, 0) {d};

\draw[-] (a) -- (b) -- (c) -- (d);
%\draw[dashed] (A) -- (C);
\end{tikzpicture}
\end{center}
come facciamo a calcolare il numero di insiemi stabili composti da 2 vertici? Sicuramente sono presenti quegli insiemi stabili di 2 vertici che otterrei togliendo un vertice, e cioè quelli ottenibili da questo cammino
\begin{center}
\begin{tikzpicture}

  \node[circle,fill=blue!20,draw, minimum size=6mm] (a) at (0, 0) {a};
  \node[circle,fill=blue!20,draw, minimum size=6mm] (b) at (2, 1) {b};
  \node[circle,fill=blue!20,draw, minimum size=6mm] (c) at (4, -0.5) {c};

\draw[-] (a) -- (b) -- (c);
%\draw[dashed] (A) -- (C);
\end{tikzpicture}
\end{center}
e cioè il solo insieme $\{a, c\}$. Quindi nel calcolo posso aggiungere $S(P_{n-1,k})$. Ma cosa accade aggiungendo un vertice? Dovrei trovare tutti i possibili insiemi stabili composti da 2 elementi, ma che non siano adiacenti al vertice aggiunto alla fine. Come posso calcolarli? Molto banalmente posso rendermi conto che se considero il cammino che non tiene presente del nodo adiacente e quello appena aggiunto,
\begin{center}
\begin{tikzpicture}

  \node[circle,fill=blue!20,draw, minimum size=6mm] (a) at (0, 0) {a};
  \node[circle,fill=blue!20,draw, minimum size=6mm] (b) at (2, 1) {b};

\draw[-] (a) -- (b);
%\draw[dashed] (A) -- (C);
\end{tikzpicture}
\end{center}

se conto i suoi insiemi stabili di 1 elemento ($S(P_{n-2,k-1})$), (cioè gli insiemi singoletto $\{a\}$ e $\{b\}$) ottengo automaticamente anche il numero di insiemi stabili da 2 elementi. Per ottenerli infatti basta aggiungere agli insiemi stabili da 1 il nuovo vertice $d$, ottenendo gli insiemi stabili validi $\{a, d\}$ e $\{b, d\}$ che corrispondono effettivamente agli insiemi stabili da 2 elementi mancanti. Possiamo quindi dedurre che nel calcolo del numero di insiemi stabili da $k$ vertici di un cammino di lunghezza arbitraria $n$ questa formula ricorsiva
\[
S(P_{n,k}) = S(P_{n-2, k-1}) + S(P_{n-1, k})
\]
è valida.
\begin{table}[H]
\centering
\begin{tabular}{l|l|ccccccccc}
$P_{n,k}$             & \multicolumn{10}{c|}{$k$}                                                                                                                                                                                                                              \\ \hline
\multirow{10}{*}{$n$} & \multicolumn{1}{c|}{} & \multicolumn{1}{c|}{0} & \multicolumn{1}{c|}{1} & \multicolumn{1}{c|}{2} & \multicolumn{1}{c|}{3} & \multicolumn{1}{c|}{4} & \multicolumn{1}{c|}{5} & \multicolumn{1}{c|}{6} & \multicolumn{1}{c|}{7} & \multicolumn{1}{c|}{8} \\ \cline{2-11} 
                      & 0                     & 1                      &                        &                        &                        &                        &                        &                        &                        &                        \\ \cline{2-2}
                      & 1                     & 1                      & 1                      &                        &                        &                        &                        &                        &                        &                        \\ \cline{2-2}
                      & 2                     & 1                      & 2                      &                        &                        &                        &                        &                        &                        &                        \\ \cline{2-2}
                      & 3                     & 1                      & 3                      & 1                      &                        &                        &                        &                        &                        &                        \\ \cline{2-2}
                      & 4                     & 1                      & 4                      & 3                      &                        &                        &                        &                        &                        &                        \\ \cline{2-2}
                      & 5                     & 1                      & 5                      & 6                      & 1                      &                        &                        &                        &                        &                        \\ \cline{2-2}
                      & 6                     & 1                      & 6                      & 10                     & 4                      &                        &                        &                        &                        &                        \\ \cline{2-2}
                      & 7                     & 1                      & 7                      & 15                     & 10                     & 1                      &                        &                        &                        &                        \\ \cline{2-2}
                      & 8                     & 1                      & 8                      & 21                     & 20                     & 5                      &                        &                        &                        &                        \\ \cline{1-2}
\end{tabular}
\caption{Numero di set indipendenti al variare del numero di vertici $n$ e della larghezza del set $k$}
\label{my-label}
\end{table}
}
\problemAnswer{ 
Per ottenere il numero totale di set indipendenti di un cammino basta effettuare la seguente somma
\[
S(P_n) = \sum_{k=0}^{n}S(P_{n,k})
\] 
Dalla tabella è interessante inoltre notare come si sia generato il triangolo di \emph{Fibonacci} (anche se non nella sua consueta forma) che è una diretta conseguenza della formula ricorsiva ottenuta.
}
\end{homeworkProblem}

\begin{homeworkProblem}[Esercizio 1.2] 
Scrivere un algoritmo capace di enumerare gli insiemi stabili di un grafo arbitrario.

\vspace{10pt} % Question

\problemAnswer{
Prima di scrivere l'algoritmo possiamo fare una considerazione: se generiamo il grafo complementare del grafo in input, come nell'esempio seguente 
\begin{center}
\begin{tikzpicture}

  \node[circle,fill=blue!20,draw, minimum size=6mm] (a) at (0, 0) {a};
  \node[circle,fill=blue!20,draw, minimum size=6mm] (b) at (2, 1) {b};
  \node[circle,fill=blue!20,draw, minimum size=6mm] (c) at (4, -0.5) {c};
  \node[circle,fill=blue!20,draw, minimum size=6mm] (d) at (6, 0) {d};

\draw[-] (a) -- (b) -- (c) -- (d);
\end{tikzpicture}
\end{center}
otteniamo questo grafo
\begin{center}
\begin{tikzpicture}

  \node[circle,fill=blue!20,draw, minimum size=6mm] (a) at (0, 0) {a};
  \node[circle,fill=blue!20,draw, minimum size=6mm] (b) at (2, 1) {b};
  \node[circle,fill=blue!20,draw, minimum size=6mm] (c) at (4, -0.5) {c};
  \node[circle,fill=blue!20,draw, minimum size=6mm] (d) at (6, 0) {d};

\draw[dashed] (a) -- (c);
\draw[dashed] (a) -- (d);
\draw[dashed] (b) -- (d);
\end{tikzpicture}
\end{center}
Se cerchiamo tutte le \emph{componenti complete} di questo grafo (sottografi del grafo i cui vertici sono tutti collegati tra loro) otteniamo esattamente ciò che stiamo cercando e cioè i nostri set indipendenti, che nell'ordine sono $\{a\}, \{b\}, \{c\}, \{d\}, \{a, c\}, \{a, d\}, \{b, d\}$ e l'insieme vuoto.


Questa proprietà del grafo complementare vale per qualsiasi tipo di grafo in input, che possiamo sfruttare per il seguente algoritmo
\begin{center}
\begin{algorithm}[H]
\SetAlgoLined
\caption{Enumerazione Set Indipendenti}
 CompGraph = \emph{getCompGraph}(Graph)\;
 TotalSet = \emph{getTotalSet}(CompGraph)\;
 \For{each set in TotalSet}{
  \If{IndipendentSet(set, CompGraph)}{
   yield set\;
   }
 }
\end{algorithm}
\end{center}
che adesso spiegheremo nel dettaglio.
}
\problemAnswer{
Per la rappresentazione del grafo può essere utile utilizzare le \emph{liste di adiacenza}. Cioè ad ognuno dei nodi viene associata una lista che contiene tutti i nodi ad esso adiacenti, come nell'esempio seguente
\begin{center}
\begin{tikzpicture}[thick,scale=0.6, every node/.style={scale=0.6}]

  \node[circle,fill=blue!20,draw, minimum size=6mm] (a) at (0, 0) {a};
  \node[circle,fill=blue!20,draw, minimum size=6mm] (b) at (2, 1) {b};
  \node[circle,fill=blue!20,draw, minimum size=6mm] (c) at (4, -0.5) {c};
  \node[circle,fill=blue!20,draw, minimum size=6mm] (d) at (6, 0) {d};

\draw[dashed] (a) -- (c);
\draw[dashed] (a) -- (d);
\draw[dashed] (b) -- (d);
\end{tikzpicture}
\end{center}
\begin{center}
\begin{tikzpicture}

  \node[draw, minimum size=5mm] (a) at (0, 0) {a};
  \node[draw, minimum size=5mm] (c) at (1, 0) {c};
  \node[draw, minimum size=5mm] (d) at (1.5, 0) {d};
  
  \node[draw, minimum size=5mm] (b) at (0, -1) {b};
  \node[draw, minimum size=5mm] (dd) at (1, -1) {d};
  
  \node[draw, minimum size=5mm] (cc) at (0, -2) {c};
  \node[draw, minimum size=5mm] (aa) at (1, -2) {a};
  
  \node[draw, minimum size=5mm] (ddd) at (0, -3) {d};
  \node[draw, minimum size=5mm] (aaa) at (1, -3) {a};
  \node[draw, minimum size=5mm] (bb) at (1.5, -3) {b};

\draw[->] (a) -- (c);
\draw[->] (b) -- (dd);
\draw[->] (cc) -- (aa);
\draw[->] (ddd) -- (aaa);
\end{tikzpicture}
\end{center}
\textbf{\emph{getCompGraph}(Graph)}

Per ottenere il grafo complementare da questa rappresentazione basta semplicemente complementare le liste (e cioè inserire nella lista vuota tutti i vertici che \textbf{non} erano presenti nel grafo originale).
\newline
\newline
\textbf{\emph{getTotalSet}(CompGraph)}

Una volta poi ottenuto il grafo complemento si cominciano ad elencare tutti i sottoinsiemi di $k$ vertici (al variare di $k$ da 0 a $n$, che è il numero totale di vertici del grafo) per un totale di
\[
\sum_{k=0}^{n}\binom{n}{k} = 2^{n}
\]
sottoinsiemi differenti.
\newline
\newline
\textbf{\textit{IndipendentSet}(set, CompGraph)}

Infine si verifica per ognuno di questi set di $k$ vertici se rappresentano un sottoinsieme di vertici completo (che corrisponde nel grafo complementare ad un set indipendente), e nel caso di esito positivo mandarlo nell'output.

Per migliorare l'algoritmo si possono fare alcune assunzioni. Come ad esempio quella che dato un grafo completo, anche tutti i suoi sottoinsiemi lo sono. Quindi un'idea potrebbe essere quella di ricercare i sottografi completi più grandi del grafo complemento, partendo quindi dagli insiemi di vertici più numerosi.
}
\end{homeworkProblem}
\begin{homeworkProblem}[Esercizio 2] 

Trovare la formula di ricorrenza di un ciclo $C_{n, k}$ al variare della lunghezza del ciclo $n$ e del numero $k$ di vertici non adiacenti.
\vspace{10pt} % Question

\problemAnswer{
Così come lo abbiamo verificato per i set indipendenti dei cammini $P_{n,k}$, possiamo fare la medesima considerazione per i cicli $C_{n,k}$. Considerando ad esempio il ciclo seguente composto da 6 vertici (e il suo grafo complementare tratteggiato)
\begin{center}
\begin{tikzpicture}[thick,scale=0.6, every node/.style={scale=0.6}]

  \node[circle,fill=blue!20,draw, minimum size=6mm] (a) at (2, 2) {a};
  \node[circle,fill=blue!20,draw, minimum size=6mm] (b) at (4, 0) {b};
  \node[circle,fill=blue!20,draw, minimum size=6mm] (c) at (4, -2) {c};
  \node[circle,fill=blue!20,draw, minimum size=6mm] (d) at (2, -4) {d};
  \node[circle,fill=blue!20,draw, minimum size=6mm] (e) at (0, -2) {e};
  \node[circle,fill=blue!20,draw, minimum size=6mm] (f) at (0, 0) {f};
  
  \draw[-] (a) -- (b);
  \draw[-] (b) -- (c);
  \draw[-] (c) -- (d);
  \draw[-] (d) -- (e);
  \draw[-] (e) -- (f);
  \draw[-] (f) -- (a);
  
  \draw[dashed] (a) -- (c);
  \draw[dashed] (a) -- (e);
  \draw[dashed] (a) -- (d);
  \draw[dashed] (a) -- (f);
  \draw[dashed] (b) -- (e);
  \draw[dashed] (e) -- (c);
  \draw[dashed] (d) -- (b);
  \draw[dashed] (f) -- (c);
  \draw[dashed] (f) -- (b);
  \draw[dashed] (f) -- (d);

%\draw[dashed] (A) -- (C);
\end{tikzpicture}
\end{center}
Se volessimo calcolare il numero di set indipendenti di 2 vertici, possiamo prima di tutto considerare il ciclo precedente a questo, privato di un vertice
\begin{center}
\begin{tikzpicture}[thick,scale=0.6, every node/.style={scale=0.6}]

  \node[circle,fill=blue!20,draw, minimum size=6mm] (a) at (2, 2) {a};
  \node[circle,fill=blue!20,draw, minimum size=6mm] (b) at (4, 0) {b};
  \node[circle,fill=blue!20,draw, minimum size=6mm] (c) at (4, -2) {c};
  \node[circle,fill=blue!20,draw, minimum size=6mm] (e) at (0, -2) {e};
  \node[circle,fill=blue!20,draw, minimum size=6mm] (f) at (0, 0) {f};
  
  \draw[-] (a) -- (b);
  \draw[-] (b) -- (c);
  \draw[-] (e) -- (f);
  \draw[-] (f) -- (a);
  
  \draw[dashed] (a) -- (c);
  \draw[dashed] (a) -- (e);
  \draw[dashed] (a) -- (f);
  \draw[dashed] (b) -- (e);
  \draw[-] (e) -- (c);
  \draw[dashed] (f) -- (c);
  \draw[dashed] (f) -- (b);

%\draw[dashed] (A) -- (C);
\end{tikzpicture}
\end{center}
Dove infatti possiamo verificare che esistono ancora alcuni dei set indipendenti di 2 vertici presenti anche nel grafo iniziale. Quindi sappiamo per certo che per calcolare $S(C_{n,k})$ dobbiamo calcolare $S(C_{n-1, k})$ + qualcosa. Questo qualcosa si ottiene dall'aggiunta di un vertice. Ma cosa accade se aggiungo un vertice? Dovrei trovare tutti gli insiemi stabili che partono dal nuovo vertice in questione, ma non considerando i due vertici adiacenti all'aggiunta ($e, c$). In tal senso quindi mi basterebbe calcolare gli insiemi stabili di $k-1$ vertici con $n-3$ vertici totali $S(n-3,k-1)$, che nel nostro caso corrisponderebbe agli insiemi stabili composti da 1 vertice ($\{f\}, \{a\}, \{b\}$), che sono 3, aggiungendo poi il nuovo vertice $d$ ad ognuno di questi ($\{f, d\}, \{a, d\}, \{b, d\}$), dovremmo ottenere i restanti set indipendenti da 2 vertici, come si evince dall'immagine seguente
\begin{center}
\begin{tikzpicture}[thick,scale=0.6, every node/.style={scale=0.6}]

  \node[circle,fill=red!20,draw, minimum size=6mm] (a) at (-2, 2) {a};
  \node[circle,fill=red!20,draw, minimum size=6mm] (b) at ( 0, 0) {b};
  \node[circle,fill=red!20,draw, minimum size=6mm] (f) at (-4, 0) {f};
  
  \node[circle,fill=blue!20,draw, minimum size=6mm] (aa) at (4, 2) {a};
  \node[circle,fill=blue!20,draw, minimum size=6mm] (bb) at ( 6, 0) {b};
  \node[circle,fill=blue!20,draw, minimum size=6mm] (ff) at (2, 0) {f};
  \node[circle,fill=blue!20,draw, minimum size=6mm] (d) at (4, -4) {d};
  
  \draw[-] (a) -- (b);
  \draw[-] (b) -- (f);
  \draw[-] (f) -- (a);
  
  \draw[-] (aa) -- (bb);
  \draw[-] (bb) -- (ff);
  \draw[-] (ff) -- (aa);
  \draw[dashed] (aa) -- (d);
  \draw[dashed] (bb) -- (d);
  \draw[dashed] (ff) -- (d);

%\draw[dashed] (A) -- (C);
\end{tikzpicture}
\end{center}
}
\problemAnswer{
Purtroppo questo calcolo non tiene conto del fatto che nell'aggiunta di un nuovo nodo, un lato che prima era presente viene poi spezzato, come mostrato nell'immagine seguente
\begin{center}
\begin{tikzpicture}[thick,scale=0.6, every node/.style={scale=0.6}]

  \node[circle,fill=blue!20,draw, minimum size=6mm] (a) at (2, 2) {a};
  \node[circle,fill=blue!20,draw, minimum size=6mm] (b) at (4, 0) {b};
  \node[circle,fill=blue!20,draw, minimum size=6mm] (c) at (4, -2) {c};
  \node[circle,fill=blue!20,draw, minimum size=6mm] (e) at (0, -2) {e};
  \node[circle,fill=blue!20,draw, minimum size=6mm] (f) at (0, 0) {f};
  
  \draw[-] (a) -- (b);
  \draw[-] (b) -- (c);
  \draw[-] (e) -- (f);
  \draw[-] (f) -- (a);
  
  \draw[dashed] (a) -- (c);
  \draw[dashed] (a) -- (e);
  \draw[dashed] (a) -- (f);
  \draw[dashed] (b) -- (e);
  \draw[-, red] (e) -- (c);
  \draw[dashed] (f) -- (c);
  \draw[dashed] (f) -- (b);
  
  \node[circle,fill=blue!20,draw, minimum size=6mm] (aa) at (8, 2) {a};
  \node[circle,fill=blue!20,draw, minimum size=6mm] (bb) at (10, 0) {b};
  \node[circle,fill=blue!20,draw, minimum size=6mm] (cc) at (10, -2) {c};
  \node[circle,fill=blue!20,draw, minimum size=6mm] (dd) at (8, -4) {d};
  \node[circle,fill=blue!20,draw, minimum size=6mm] (ee) at (6, -2) {e};
  \node[circle,fill=blue!20,draw, minimum size=6mm] (ff) at (6, 0) {f};
  
  \draw[-] (aa) -- (bb);
  \draw[-] (bb) -- (cc);
  \draw[-] (cc) -- (dd);
  \draw[-] (dd) -- (ee);
  \draw[-] (ee) -- (ff);
  \draw[-] (ff) -- (aa);
  
  \draw[dashed] (aa) -- (cc);
  \draw[dashed] (aa) -- (ee);
  \draw[dashed] (aa) -- (dd);
  \draw[dashed] (aa) -- (ff);
  \draw[dashed] (bb) -- (ee);
  \draw[dashed, red] (ee) -- (cc);
  \draw[dashed] (dd) -- (bb);
  \draw[dashed] (ff) -- (cc);
  \draw[dashed] (ff) -- (bb);
  \draw[dashed] (ff) -- (dd);

%\draw[dashed] (A) -- (C);
\end{tikzpicture}
\end{center}
che rischierebbe così di non essere contato. Per questa motivazione, invece di considerare il passo $n-3$ nel calcolo degli elementi rimanenti, viene considerato il passo $n-2$ che conta invece 4 set indipendenti da 1 vertice, spezzando un lato che prima era presente (come abbiamo mostrato prima per l'aggiunta di un vertice), e che permette quindi di completare la formula ricorsiva
\[
S(C_{n,k}) = S(C_{n-1,k}) + S(C_{n-2, k-1})
\]
}
\end{homeworkProblem}

\begin{homeworkProblem}[Esercizio 2.1] 

Tabella degli insiemi stabili di un ciclo $C_{n,k}$ al variare della lunghezza del ciclo $n$ e del numero $k$ di vertici non adiacenti.
\vspace{10pt} % Question

\problemAnswer{ % Answer
\begin{table}[H]
\centering
\begin{tabular}{l|l|ccccccccc}
$C_{n,k}$             & \multicolumn{10}{c|}{$k$}                                                                                                                                                                                                                              \\ \hline
\multirow{10}{*}{$n$} & \multicolumn{1}{c|}{} & \multicolumn{1}{c|}{0} & \multicolumn{1}{c|}{1} & \multicolumn{1}{c|}{2} & \multicolumn{1}{c|}{3} & \multicolumn{1}{c|}{4} & \multicolumn{1}{c|}{5} & \multicolumn{1}{c|}{6} & \multicolumn{1}{c|}{7} & \multicolumn{1}{c|}{8} \\ \cline{2-11} 
                      & 0                     & 1                      &                        &                        &                        &                        &                        &                        &                        &                        \\ \cline{2-2}
                      & 1                     & 1                      & 1                      &                        &                        &                        &                        &                        &                        &                        \\ \cline{2-2}
                      & 2                     & 1                      & 2                      &                        &                        &                        &                        &                        &                        &                        \\ \cline{2-2}
                      & 3                     & 1                      & 3                      &                        &                        &                        &                        &                        &                        &                        \\ \cline{2-2}
                      & 4                     & 1                      & 4                      & 2                      &                        &                        &                        &                        &                        &                        \\ \cline{2-2}
                      & 5                     & 1                      & 5                      & 5                      &                        &                        &                        &                        &                        &                        \\ \cline{2-2}
                      & 6                     & 1                      & 6                      & 9                      & 2                      &                        &                        &                        &                        &                        \\ \cline{2-2}
                      & 7                     & 1                      & 7                      & 14                     & 7                      &                        &                        &                        &                        &                        \\ \cline{2-2}
                      & 8                     & 1                      & 8                      & 20                     & 16                     & 2                      &                        &                        &                        &                        \\ \cline{1-2}
\end{tabular}
\caption{Numero di set indipendenti da $k$ vertici in un ciclo di $n$ elementi.}
\label{my-label}
\end{table}
}
\end{homeworkProblem}
%----------------------------------------------------------------------------------------

\end{document}
