%%%%%%%%%%%%%%%%%%%%%%%%%%%%%%%%%%%%%%%%%
% Structured General Purpose Assignment
% LaTeX Template
%
% This template has been downloaded from:
% http://www.latextemplates.com
%
% Original author:
% Ted Pavlic (http://www.tedpavlic.com)
%
% Note:
% The \lipsum[#] commands throughout this template generate dummy text
% to fill the template out. These commands should all be removed when 
% writing assignment content.
%
%%%%%%%%%%%%%%%%%%%%%%%%%%%%%%%%%%%%%%%%%

%----------------------------------------------------------------------------------------
%	PACKAGES AND OTHER DOCUMENT CONFIGURATIONS
%----------------------------------------------------------------------------------------

\documentclass{article}

\usepackage{fancyhdr} % Required for custom headers
\usepackage{lastpage} % Required to determine the last page for the footer
\usepackage{extramarks} % Required for headers and footers
\usepackage{graphicx} % Required to insert images
\usepackage{lipsum} % Used for inserting dummy 'Lorem ipsum' text into the template
\usepackage{amsmath}
\usepackage{algorithm2e}
\usepackage[italian]{babel}
\usepackage[utf8x]{inputenc}
\usepackage{array}
\usepackage{float}
\usepackage{tikz}
\usepackage{multirow}
\usepackage[usestackEOL]{stackengine}[2013-10-15]
\def\x{\hspace{3ex}}    %BETWEEN TWO 1-DIGIT NUMBERS
\def\y{\hspace{2.45ex}}  %BETWEEN 1 AND 2 DIGIT NUMBERS
\def\z{\hspace{1.9ex}}    %BETWEEN TWO 2-DIGIT NUMBERS
\stackMath

% Margins
\topmargin=-0.45in
\evensidemargin=0in
\oddsidemargin=0in
\textwidth=6.5in
\textheight=9.0in
\headsep=0.25in 

\linespread{1.1} % Line spacing

% Set up the header and footer
\pagestyle{fancy}
\lhead{\hmwkAuthorName} % Top left header
\chead{\hmwkClass\: \hmwkTitle} % Top center header
\rhead{\firstxmark} % Top right header
\lfoot{\lastxmark} % Bottom left footer
\cfoot{} % Bottom center footer
\rfoot{Page\ \thepage\ of\ \pageref{LastPage}} % Bottom right footer
\renewcommand\headrulewidth{0.4pt} % Size of the header rule
\renewcommand\footrulewidth{0.4pt} % Size of the footer rule

\setlength\parindent{0pt} % Removes all indentation from paragraphs

%----------------------------------------------------------------------------------------
%	DOCUMENT STRUCTURE COMMANDS
%	Skip this unless you know what you're doing
%----------------------------------------------------------------------------------------

% Header and footer for when a page split occurs within a problem environment
\newcommand{\enterProblemHeader}[1]{
\nobreak\extramarks{#1}{#1 continued on next page\ldots}\nobreak
\nobreak\extramarks{#1 (continued)}{#1 continued on next page\ldots}\nobreak
}

% Header and footer for when a page split occurs between problem environments
\newcommand{\exitProblemHeader}[1]{
\nobreak\extramarks{#1 (continued)}{#1 continued on next page\ldots}\nobreak
\nobreak\extramarks{#1}{}\nobreak
}

\setcounter{secnumdepth}{0} % Removes default section numbers
\newcounter{homeworkProblemCounter} % Creates a counter to keep track of the number of problems

\newcommand{\homeworkProblemName}{}
\newenvironment{homeworkProblem}[1][Problem \arabic{homeworkProblemCounter}]{ % Makes a new environment called homeworkProblem which takes 1 argument (custom name) but the default is "Problem #"
\stepcounter{homeworkProblemCounter} % Increase counter for number of problems
\renewcommand{\homeworkProblemName}{#1} % Assign \homeworkProblemName the name of the problem
\section{\homeworkProblemName} % Make a section in the document with the custom problem count
\enterProblemHeader{\homeworkProblemName} % Header and footer within the environment
}{
\exitProblemHeader{\homeworkProblemName} % Header and footer after the environment
}

\newcommand{\problemAnswer}[1]{ % Defines the problem answer command with the content as the only argument
\noindent\framebox[\columnwidth][c]{\begin{minipage}{0.98\columnwidth}#1\end{minipage}} % Makes the box around the problem answer and puts the content inside
}

\newcommand{\homeworkSectionName}{}
\newenvironment{homeworkSection}[1]{ % New environment for sections within homework problems, takes 1 argument - the name of the section
\renewcommand{\homeworkSectionName}{#1} % Assign \homeworkSectionName to the name of the section from the environment argument
\subsection{\homeworkSectionName} % Make a subsection with the custom name of the subsection
\enterProblemHeader{\homeworkProblemName\ [\homeworkSectionName]} % Header and footer within the environment
}{
\enterProblemHeader{\homeworkProblemName} % Header and footer after the environment
}
   
%----------------------------------------------------------------------------------------
%	NAME AND CLASS SECTION
%----------------------------------------------------------------------------------------

\newcommand{\hmwkTitle}{Esercizi (Partizioni Stabili)} % Assignment title
\newcommand{\hmwkDueDate}{Monday,\ January\ 1,\ 2012} % Due date
\newcommand{\hmwkClass}{Teoria Dei Grafi} % Course/class
\newcommand{\hmwkClassTime}{} % Class/lecture time
\newcommand{\hmwkClassInstructor}{Prof. Ottavio D'Antona} % Teacher/lecturer
\newcommand{\hmwkAuthorName}{Marco Odore} % Your name
\newcommand{\hmwkMatricola}{868906}

%----------------------------------------------------------------------------------------
%	TITLE PAGE
%----------------------------------------------------------------------------------------

\title{
\vspace{2in}
\textmd{\textbf{\hmwkClass:\ \hmwkTitle}}\\
\vspace{0.1in}\large{\textit{\hmwkClassInstructor\ \hmwkClassTime}}
\vspace{3in}
}

\author{\textbf{\hmwkAuthorName\ \hmwkMatricola}}
\date{} % Insert date here if you want it to appear below your name

%----------------------------------------------------------------------------------------

\begin{document}

\maketitle

%----------------------------------------------------------------------------------------
%	TABLE OF CONTENTS
%----------------------------------------------------------------------------------------

%\setcounter{tocdepth}{1} % Uncomment this line if you don't want subsections listed in the ToC

\newpage
\tableofcontents
\newpage

%----------------------------------------------------------------------------------------
%	PROBLEM 1
%----------------------------------------------------------------------------------------

% To have just one problem per page, simply put a \clearpage after each problem

\begin{homeworkProblem}[Esercizio 1] 
Dimostrare che il numero di partizioni stabili di un path di $n$ elementi, è uguale al numero di partizioni di un insieme di $n-1$ elementi tramite biiezione.
\newline
\newline
\textbf{N.B.} Le partizioni stabili sono quelle partizioni che generano blocchi in cui i vertici non sono adiacenti.
\vspace{10pt} % Question
\problemAnswer{ % Answer
Prendiamo ad esempio il sequente path:
\begin{center}
\begin{tikzpicture}

  \node[circle,fill=blue!20,draw, minimum size=6mm] (1) at (0, 0) {1};
  \node[circle,fill=blue!20,draw, minimum size=6mm] (2) at (2, 1) {2};
  \node[circle,fill=blue!20,draw, minimum size=6mm] (3) at (4, -0.5) {3};

\draw[-] (1) -- (2) -- (3);
\end{tikzpicture}
\end{center}
Quest'ultimo genera le seguenti partizioni stabili:
\[
|1|2|3|
\]
\[
|1 3|2|
\]
Se consideriamo l'insieme di $n-1$ elementi, parallelo al precedente cammino e cioè $\{1, 2\}$, possiamo verificare che possiede le seguenti partizioni:
\[
|1|2|
\]
\[
|1  2|
\]
Notiamo che il numero di partizioni stabili del path di $n=3$ elementi è effettivamente uguale al numero di partizioni dell'insieme composto da $n-1$ elementi (hanno entrambi 2 partizioni.)
Come passare da una partizione all'altra? Prima di tutto useremo per comodità la notazione dei numeri naturali per i vertici del path e per gli elementi dell'insieme. Per passare dalle partizioni stabili di un cammino a quelle dell'insieme si possono eseguire 4 passaggi:
\begin{enumerate}
\item Si ordinano le partizioni in base all'elemento più piccolo presente nella partizione.
\item Se l'elemento $n$ ($n$ inteso come etichetta numerica associata al vertice) si trova in una partizione più grande nell'ordinamento (ordinamento definito nel punto 1), rispetto all'elemento $n-1$, si sposta l'elemento nella partizione inferiore.
\item Si rinominano le etichette dei numeri, sottraendovi 1
\item Si eliminano i nodi con etichetta 0
\end{enumerate}
\begin{center}
\begin{tikzpicture}

  \node[circle,fill=blue!20,draw, minimum size=6mm] (1) at (0, 0) {1};
  \node[circle,fill=blue!20,draw, minimum size=6mm] (2) at (2, 1) {2};
  \node[circle,fill=blue!20,draw, minimum size=6mm] (3) at (4, -0.5) {3};
  \node[circle,fill=blue!20,draw, minimum size=6mm] (4) at (6, 0) {4};

\draw[-] (1) -- (2) -- (3) -- (4);
%\draw[dashed] (A) -- (C);
\end{tikzpicture}
\end{center}
}
\end{homeworkProblem}

%----------------------------------------------------------------------------------------

\end{document}
